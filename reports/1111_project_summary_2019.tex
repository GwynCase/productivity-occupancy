\documentclass[]{article}
\usepackage{lmodern}
\usepackage{amssymb,amsmath}
\usepackage{ifxetex,ifluatex}
\usepackage{fixltx2e} % provides \textsubscript
\ifnum 0\ifxetex 1\fi\ifluatex 1\fi=0 % if pdftex
  \usepackage[T1]{fontenc}
  \usepackage[utf8]{inputenc}
\else % if luatex or xelatex
  \ifxetex
    \usepackage{mathspec}
  \else
    \usepackage{fontspec}
  \fi
  \defaultfontfeatures{Ligatures=TeX,Scale=MatchLowercase}
\fi
% use upquote if available, for straight quotes in verbatim environments
\IfFileExists{upquote.sty}{\usepackage{upquote}}{}
% use microtype if available
\IfFileExists{microtype.sty}{%
\usepackage[]{microtype}
\UseMicrotypeSet[protrusion]{basicmath} % disable protrusion for tt fonts
}{}
\PassOptionsToPackage{hyphens}{url} % url is loaded by hyperref
\usepackage[unicode=true]{hyperref}
\hypersetup{
            pdftitle={2019 Project Summary},
            pdfborder={0 0 0},
            breaklinks=true}
\urlstyle{same}  % don't use monospace font for urls
\usepackage[margin=1in]{geometry}
\usepackage{longtable,booktabs}
% Fix footnotes in tables (requires footnote package)
\IfFileExists{footnote.sty}{\usepackage{footnote}\makesavenoteenv{long table}}{}
\usepackage{graphicx,grffile}
\makeatletter
\def\maxwidth{\ifdim\Gin@nat@width>\linewidth\linewidth\else\Gin@nat@width\fi}
\def\maxheight{\ifdim\Gin@nat@height>\textheight\textheight\else\Gin@nat@height\fi}
\makeatother
% Scale images if necessary, so that they will not overflow the page
% margins by default, and it is still possible to overwrite the defaults
% using explicit options in \includegraphics[width, height, ...]{}
\setkeys{Gin}{width=\maxwidth,height=\maxheight,keepaspectratio}
\IfFileExists{parskip.sty}{%
\usepackage{parskip}
}{% else
\setlength{\parindent}{0pt}
\setlength{\parskip}{6pt plus 2pt minus 1pt}
}
\setlength{\emergencystretch}{3em}  % prevent overfull lines
\providecommand{\tightlist}{%
  \setlength{\itemsep}{0pt}\setlength{\parskip}{0pt}}
\setcounter{secnumdepth}{5}
% Redefines (sub)paragraphs to behave more like sections
\ifx\paragraph\undefined\else
\let\oldparagraph\paragraph
\renewcommand{\paragraph}[1]{\oldparagraph{#1}\mbox{}}
\fi
\ifx\subparagraph\undefined\else
\let\oldsubparagraph\subparagraph
\renewcommand{\subparagraph}[1]{\oldsubparagraph{#1}\mbox{}}
\fi

% set default figure placement to htbp
\makeatletter
\def\fps@figure{htbp}
\makeatother

\usepackage{fontspec} \setmainfont{Sitka Display} \newfontfamily\ssf{Lato} \newenvironment{ctable}{\ssf }{} \newenvironment{capctable}[1][t]{\begin{table}[#1]\centering\ssf}{\end{table}}
\usepackage{booktabs}
\usepackage{longtable}
\usepackage{array}
\usepackage{multirow}
\usepackage{wrapfig}
\usepackage{float}
\usepackage{colortbl}
\usepackage{pdflscape}
\usepackage{tabu}
\usepackage{threeparttable}
\usepackage{threeparttablex}
\usepackage[normalem]{ulem}
\usepackage{makecell}
\usepackage{xcolor}

\title{2019 Project Summary}
\author{}
\date{\vspace{-2.5em}}

\begin{document}
\maketitle

\section{Abstract}\label{abstract}

The Northern goshawk is an at-risk species in coastal BC which lives and
breeds in mature and old-growth forests. Current goshawk management
focuses on protecting habitat for breeding, and a goshawk's requirements
for breeding habitat are well understood. However, habitat used for
foraging, while acknowledged to also be important, is largely excluded
from management plans. This is primarily due to a lack of knowledge
about foraging habitat requirements, particularly regarding how foraging
habitat differs from breeding habitat and how much foraging habitat is
needed.

In 2018 I began a

\section{Diet}\label{diet}

Introductory paragraph.

\subsection{Physical specimens}\label{physical-specimens}

Physical remains were collected using two different methodologies.
Opportunistic collections were gathered by inventory technicians during
regular goshawk surveys. Prey remains and regurgiated pellets were
collected from beneath pluck posts, perches, and active and inactive
nests when discovered by surveyors. Items from each pluck post, perch,
or nest were pooled into a single sample. Systematic collections were
gathered during thorough searches of the ground within a 50-m radius of
an active nest. All physical remains from a single nest area search were
pooled into a single sample.

We reconstructed physical remains following a modification of Lewis et
al. (2006). Within each sample, prey remains were identified to the
lowest possible taxonomic category and the minimum number of individuals
counted (ie 1.5 vole mandibles = 2 voles {[}family: Cricetidae{]}).
Intact and broken but reassembled pellets were analyzed individually
within each sample, while fragmented pellets were combined within each
sample. Pellets were dissected and feathers, fur, and hard parts (bones,
teeth, claws) were identified to the lowest taxonomic level. We counted
the minimum number of individuals represented within the pellet or
pellet collection (ie, Douglas squirrel fur and 3 squirrel claws = 1
\emph{Tamiascuirus douglasii}). Items were additionally categorized to
size and assigned mass as per camera data (see below).

\subsubsection{Preliminary results}\label{preliminary-results}

We collected prey remains and pellets from 15 sites during the 2019
breeding season. At least 7 sites have one or more systematic
collection. We have identified 28 prey items so far, 43\% to family and
25\% to genus or species.

\begin{capctable}

\caption{\label{tab:unnamed-chunk-2}Prey identified from physical remains}
\centering
\begin{tabular}[t]{llllr}
\toprule
Class & Family & Genus & Species & N\\
\midrule
Aves & Columbidae & Patagioenas & fasciata & 1\\
Aves & Parulidae & U & U & 1\\
Aves & Passerellidae & Pipilo & maculatus & 1\\
Aves & Passerellidae & U & U & 1\\
Aves & Picidae & Colaptes & auratus & 3\\
\addlinespace
Aves & U & U & U & 11\\
Mammalia & Cricetidae & U & sp & 2\\
Mammalia & Rodentidae & U & U & 1\\
Mammalia & Sciuridae & Tamiasciurus & douglasii & 1\\
Mammalia & Scuiridae & Tamiasciurus & douglasii & 1\\
\addlinespace
Mammalia & U & U & U & 5\\
\bottomrule
\end{tabular}
\end{capctable}

Preliminary results indicate that physical remains may be more accurate
for identifying small mammals and birds, but less accurate for
identifying large birds. Multiple species not recorded in cameras were
identified using physical remains.

Further analysis is currently on hold due to the covid-19 pandemic.

\subsection{Cameras}\label{cameras}

\subsubsection{Methods}\label{methods}

We quantified the diet of breeding goshawks using digital trail cameras
placed at 6 nests during 2019. Cameras were programmed to take three
photos one second apart when triggered by motion, and an additional one
photo every thirty minutes. Installation took place during the early
nestling phase (between 4 June and 26 June) and cameras were left in
place until after juvenile dispersal.

Nest camera photos were reviewed and each new prey item was recorded and
identified to species when possible. When identification to species was
not possible, items were identified to the lowest possible taxonomic
level. Items were additionally categorized by size (small, medium, or
large). Prey items identified to species were assigned mass using data
from the literature. Unidentified items and partial items were assigned
mass by averaging the masses of the identified species in that size and
taxonomic group.

We calculated the relative proportions of avian and mammalian biomass
delivered to all nests during the study period. For each nest, we
calculated the mean prey deliveries per day by count and by biomass.
Daily biomass for all six nests was pooled to determine the effect of
brood size and brood age on delivery rate.

We calculated prey species diversity for the entire study area and for
each nest using items identified to genus or species using Simpson's
Diversity Index. We calculated dietary overlap between nests using
Morisita's Index of Similarity.

\subsubsection{Preliminary results}\label{preliminary-results-1}

We observed no nest abandonment following camera installation. One nest
failed 9 days following camera install, while the other five nests
successfully fledged at least one chick. Successfully nests fledged 1 (n
= 1), 2 (n = 3), or 3 (n = 1) chicks. The unsuccessful nest failed after
two chicks succumbed to siblicide and the third appeared to fledge
prematurely, though the exact cause of failure is unknown. Two other
nests were observed to lose a single chick each due to apparent
siblicide.

We obtained 26577 photos from 6 nests during the 2019 breeding season. A
total of 268 prey item deliveries were recorded. 16\% of items were
obscured from the camera during delivery and consumption and were
removed from the analysis. Out of the 225 visible items, 75\% were
identified to class and 59\% to genus or species. Small and medium birds
were disproportionately represented among unidentified items, frequently
arriving at the nest already plucked and decapitated.

\begin{capctable}

\caption{\label{tab:unnamed-chunk-3}Deliveries recorded on nest cameras}
\centering
\begin{tabular}[t]{lllrr}
\toprule
Site & First delivery recorded & Last delivery recorded & N. deliveries & Deliveries/day\\
\midrule
MTC & 2019-06-11 & 2019-07-21 & 51 & 1\\
MTF & 2019-06-04 & 2019-07-08 & 73 & 2\\
RLK & 2019-06-22 & 2019-07-15 & 20 & 1\\
TCR & 2019-06-10 & 2019-07-08 & 44 & 2\\
TMC & 2019-06-17 & 2019-07-02 & 47 & 3\\
\addlinespace
UTZ & 2019-06-26 & 2019-07-08 & 33 & 3\\
\bottomrule
\end{tabular}
\end{capctable}

Across the entire study area, we observed 17 different prey species
delivered to nests. By biomass, mammals made up the largest proportion
of deliveries (69\%). This was due to the overwhelming number of tree
squirrels (\emph{Tamiasciurus} spp.) delivered to nests, which provided
45\% of biomass. Birds made up 17\% of the diet, with the final 14\% of
prey biomass unable to be identifed as either bird or mammal.

Prey deliveries averaged 1.86 ± 0.8 deliveries/day, although all nests
observed occasionally went at least one or more days without any
deliveries. Average biomass of items was 170.18 ± 181.64 g, and the
average daily biomass of prey delivered to each nest was 290.57 ± 191.48
g/day. Brood age did not affect delivery rates, with older broods
receiving approximately the same biomass per day as younger broods (P =
0.48).

The overall index of diversity for the diet of all 6 nests was 0.67. For
individual nests, diversity ranged from 0.74 to 0.24 (mean = 0.56).
Overlap was consistently low, ranging from 0.59 to 0.03 (mean = 0.26).

\section{Movement}\label{movement}

Introductory paragraph

\subsection{Telemetry}\label{telemetry}

\subsubsection{Methods}\label{methods-1}

\begin{capctable}

\caption{\label{tab:unnamed-chunk-4}Summary of telemetry location data}
\centering
\begin{tabular}[t]{llllr}
\toprule
Site & ID & First location & Last location & N. points\\
\midrule
MTC & HAR09 & 2019-06-11 & 2019-07-02 & 1216\\
MTC & HAR10 & 2019-06-13 & 2019-06-29 & 923\\
RLK & HAR04 & 2019-06-22 & 2019-07-08 & 1597\\
SKA & HAR05 & 2019-06-23 & 2019-09-04 & 4671\\
TCR & HAR07 & 2018-07-08 & 2019-05-08 & 2103\\
\addlinespace
TCR & HAR08 & 2019-06-10 & 2019-06-27 & 135\\
\bottomrule
\multicolumn{5}{l}{\textit{Note: }}\\
\multicolumn{5}{l}{HAR07 died sometime during winter 18-19. No points have been retrieved from the UTZ bird yet.}\\
\end{tabular}
\end{capctable}

During 2018-2019 we captured and tagged 7 adult goshawks (4 female and 3
male) at 5 active nest sites. Trapping took place during the
mid-breeding season (May-June) using a dho-gaza trap with a live
great-horned owl (\emph{Bubo virginianus}) as a lure. We fitted goshawks
with a solar-powered GPS-UHF transmitter with an additional attached VHF
transmitter. Transmitters were programmed to record a location every 15
minutes during the breeding season (approximately May-August) and every
4 hours during the nonbreeding season. Location data were retrieved from
the tag via either a base station placed near the nest or a hand-held
UHF receiver.

We used location points from 11 May to 1 September to calculate 95\% MCP
breeding season home ranges. We attempted to use hidden Markov models to
identify behavioral states from movement data. At sites with both
appropriate telemetry and camera data (\emph{n} = 2), we also attempted
to match foraging locations with prey deliveries.

We modelled predictors of nighttime roost sites and their similarity to
nest sites using several habitat variables and two habitat models. We
located roost sites using the location point taken closest to midnight
for each site. Landscape variables (canopy cover, stand age, and stand
basal area) were taken from the BC VRI and modelled habitat values from
the 2008 foraging HSI and nesting HSI.

\subsubsection{Preliminary results}\label{preliminary-results-2}

The mean breeding season homerange was 3571.05 ha, but there was a large
difference between male and female homerange size. Male homeranges
averaged 5482.58 \(\pm\) 2329.6 ha, while female home ranges averaged
1659.53 \(\pm\) 2816.12ha.

Distinguishing different behavioral states from location data has been
challenging so far. Hidden Markov models, the most widely used method,
do not appear effective at differentiating foraging locations from
travel locations. However, several alternate methods remain to be
explored. Linking deliveries observed on nest cameras to foraging
location recorded by telemetry appears more promising, but more work is
needed.

We identified 101 nighttime roost locations at 3 sites. The most
informative model (R\^{}2 = 0.16) included both canopy closure and the
nesting HSI. This model was better than the second-best informative
model by 2.72 AIC units, and better than the nesting HSI-alone model
(R\^{}2 = 0.09) by 7.90 AIC units.

\section{Occupancy \& Landscape}\label{occupancy-landscape}

It has been suggested that high-quality habitat will be occupied more
frequently than low-quality habitat. As an additional metric of habitat
quality, I plan to examine links between landscape variables identifed
in previous analyses and the historic occupancy of sites on Vancouver
Island and the South Coast. Landscape variables found to be significant
predictors of productivity or diet may also be linked to site occupancy.
This is dependent on obtaining a sufficient sample size of nests. There
are 9 Vancouver Island sites with occupancy data for both 2018 and 2019,
and 28 South Coast sites with occupancy data for the same period.

\section{Going forward}\label{going-forward}

\begin{itemize}
\tightlist
\item
  measure landscape metrics for x nests with cameras, y with pellets and
  z with occupancy data ??
\item
  complete pellet id (x nests ca y pellets per nest?)
\item
  assess link between landscape and diet
\item
  compare diet assessed using two approaches
\item
  assess link between landscape and occupancy (?)
\item
  telemetry ?? UD during breeding season, other basic descriptors that
  layout radius for landscape metrics, habitat selection ??
\end{itemize}

\section{Appendix}\label{appendix}

\begin{itemize}
\tightlist
\item
  Summary of data you have (or are using), sample sizes, what data do
  you expect to get this field season. This can be a simple table.
\end{itemize}

\newgeometry{margin=1cm}

\begin{landscape}\begin{capctable}

\caption{\label{tab:unnamed-chunk-7}Summary of available data}
\centering
\resizebox{\linewidth}{!}{
\begin{tabular}[t]{lllllllllllllll}
\toprule
\multicolumn{2}{c}{ } & \multicolumn{2}{c}{Type} & \multicolumn{2}{c}{Diet} & \multicolumn{4}{c}{Telemetry} & \multicolumn{5}{c}{Occupancy} \\
\cmidrule(l{3pt}r{3pt}){3-4} \cmidrule(l{3pt}r{3pt}){5-6} \cmidrule(l{3pt}r{3pt}){7-10} \cmidrule(l{3pt}r{3pt}){11-15}
Site & Name & Int & Ext & Phys. rem. & Cameras & 2018 M & 2018 F & 2019 M & 2019 F & 2015 & 2016 & 2017 & 2018 & 2019\\
\midrule
\rowcolor{gray!6}  UTZ & Utzilus & x &  & x & x &  &  &  & x &  &  &  &  & x\\
MTC & Mt. Currie & x &  & x & x &  &  & x & x &  &  &  &  & x\\
\rowcolor{gray!6}  MTF & Mt. Ford & x &  & x & x &  &  &  &  &  &  & x & x & x\\
RLK & Ruby Lake & x &  & x & x &  &  &  & x & x & x & x & x & x\\
\rowcolor{gray!6}  TCR & Turbid Creek & x &  & x & x & x &  &  & x & x &  & x & x & x\\
\addlinespace
TMC & Twenty-Mile Creek & x &  & x & x &  &  &  &  &  & x & x & x & x\\
\rowcolor{gray!6}  SKA & Skaiakos & x &  & x &  &  &  & x &  &  &  &  & x & x\\
PTC & Potlatch &  & x & x &  &  &  &  &  & x & x & x & x & x\\
\rowcolor{gray!6}  CSK & Comsock &  & x & x &  &  &  &  &  &  &  &  &  & \\
MPT & Middle Point &  & x & x &  &  &  &  &  &  &  &  &  & x\\
\addlinespace
\rowcolor{gray!6}  PCR & Peers Creek &  & x & x &  &  &  &  &  &  &  &  &  & x\\
BKH & Birkenhead &  & x & x &  &  &  &  &  &  &  &  &  & x\\
\rowcolor{gray!6}  WCR & Wray Creek &  & x & x &  &  &  &  &  &  &  &  &  & x\\
PNC & Pinecone &  & x & x &  &  &  &  &  &  &  &  &  & x\\
\rowcolor{gray!6}  DCR & Douglas Creek &  & x & x &  &  &  &  &  &  &  &  &  & x\\
\bottomrule
\end{tabular}}
\end{capctable}
\end{landscape}

\restoregeometry

\begin{capctable}

\caption{\label{tab:unnamed-chunk-8}Summary of nest camera data}
\centering
\begin{tabu} to \linewidth {>{\raggedright\arraybackslash}p{20em}>{\em\raggedright\arraybackslash}p{20em}>{\raggedleft}X>{\raggedleft}X>{\raggedleft}X}
\toprule
Prey species &  & count & \% count & \% biomass\\
\midrule
\addlinespace[0.3em]
\multicolumn{5}{l}{\textbf{Large birds (> 150 g)}}\\
\hspace{1em}ruffed grouse & Bonasa umbellus & 1 & 0.44 & 1.37\\
\hspace{1em}sooty grouse & Dendragapus fulignosus & 2 & 0.89 & 5.51\\
\hspace{1em}band-tailed pigeon & Patagoienas fasciata & 3 & 1.33 & 2.97\\
\addlinespace[0.3em]
\multicolumn{5}{l}{\textbf{Medium birds (60-150 g)}}\\
\hspace{1em}Steller's jay & Cyanocitta stelleri & 2 & 0.89 & 0.67\\
\hspace{1em}varied thrush & Ixoreus naevius & 7 & 3.11 & 1.45\\
\hspace{1em}gray jay & Perisoreus canadensis & 2 & 0.89 & 0.37\\
\hspace{1em}American robin & Turdus migratorius & 3 & 1.33 & 0.63\\
\hspace{1em}average medium bird &  & 11 & 4.89 & 2.44\\
\addlinespace[0.3em]
\multicolumn{5}{l}{\textbf{Small birds (< 40 g)}}\\
\hspace{1em}Swainson's thrush & Catharus ustulatus & 5 & 2.22 & 0.39\\
\hspace{1em}average small bird &  & 18 & 8.00 & 1.40\\
\addlinespace[0.3em]
\multicolumn{5}{l}{\textbf{Large mammals (> 600 g)}}\\
\hspace{1em}snowshoe hare & Lepus americanus & 2 & 0.89 & 7.00\\
\hspace{1em}average large mammal &  & 1 & 0.44 & 3.50\\
\addlinespace[0.3em]
\multicolumn{5}{l}{\textbf{Medium mammals (200-600 g)}}\\
\hspace{1em}bushy-tailed woodrat & Cricetidae cinerea & 1 & 0.44 & 0.98\\
\hspace{1em}rat & Rattus sp & 11 & 4.89 & 7.75\\
\hspace{1em}Douglas squirrel & Tamiasciurus douglasii & 73 & 32.44 & 38.80\\
\hspace{1em}red squirrel & Tamiasciurus hudsonicus & 9 & 4.00 & 5.28\\
\hspace{1em}tree squirrel & Tamiasciurus sp & 1 & 0.44 & 0.56\\
\hspace{1em}average medium mammal &  & 4 & 1.78 & 2.25\\
\addlinespace[0.3em]
\multicolumn{5}{l}{\textbf{Small mammals (< 200 g)}}\\
\hspace{1em}flying squirrel & Glaucomys sabrinus & 3 & 1.33 & 1.22\\
\hspace{1em}bat & Myotis sp & 1 & 0.44 & 0.02\\
\hspace{1em}chipmunk & Neotamias sp & 6 & 2.67 & 1.04\\
\hspace{1em}average small mammal &  & 3 & 1.33 & 0.68\\
\addlinespace[0.3em]
\multicolumn{5}{l}{\textbf{Unidentified items}}\\
\hspace{1em}average medium item &  & 13 & 5.78 & 6.00\\
\hspace{1em}average small item &  & 43 & 19.11 & 7.74\\
\bottomrule
\end{tabu}
\end{capctable}

\begin{capctable}

\caption{\label{tab:unnamed-chunk-9}Vancouver Island occupancy data}
\centering
\begin{tabular}[t]{lrr}
\toprule
Site & 2018 &  2019\\
\midrule
\rowcolor{gray!6}  Goose Creek & 1 & 1\\
Rona Loop & 1 & 1\\
\rowcolor{gray!6}  Lukwa South & 1 & 1\\
Tsitika West & 1 & 1\\
\rowcolor{gray!6}  Cook Creek & 1 & 2\\
\addlinespace
Mahatta & 1 & 1\\
\rowcolor{gray!6}  China Beach & 2 & 1\\
Taylor River & 1 & 1\\
\rowcolor{gray!6}  Keta & 3 & 1\\
\bottomrule
\multicolumn{3}{l}{\textit{Note: }}\\
\multicolumn{3}{l}{1 = no birds observed, 2 = birds present, no evidence of breeding, 3 = active, evidence of breeding}\\
\end{tabular}
\end{capctable}

\begin{capctable}

\caption{\label{tab:unnamed-chunk-10}South Coast occupancy data}
\centering
\begin{tabular}[t]{lrr}
\toprule
Site & 2018 &  2019\\
\midrule
\rowcolor{gray!6}  Dewdney Creek & 3 & 1\\
Ford Mountain & 1 & 3\\
\rowcolor{gray!6}  Mt Holden & 1 & 1\\
Silver & 2 & 1\\
\rowcolor{gray!6}  Clowhom & 3 & 3\\
\addlinespace
Duck Lake & 1 & 1\\
\rowcolor{gray!6}  Freil & 2 & 3\\
Giovanno & 2 & 1\\
\rowcolor{gray!6}  Granite Mt & 1 & 3\\
Haslam & 1 & 1\\
\addlinespace
\rowcolor{gray!6}  Maurell Island & 3 & 3\\
McNair & 1 & 2\\
\rowcolor{gray!6}  Mt Pearkes & 1 & 1\\
Nanton & 1 & 1\\
\rowcolor{gray!6}  Osgood & 1 & 3\\
\addlinespace
Phantom & 3 & 3\\
\rowcolor{gray!6}  Potlatch & 3 & 2\\
Powell Daniels & 3 & 3\\
\rowcolor{gray!6}  Ruby Lake & 2 & 3\\
Skaiakos & 1 & 3\\
\addlinespace
\rowcolor{gray!6}  St. Vincent & 1 & 3\\
Brohm & 1 & 1\\
\rowcolor{gray!6}  Dipper Creek & 1 & 3\\
Jarvis & 1 & 1\\
\rowcolor{gray!6}  Lillooette & 1 & 1\\
\addlinespace
Millars Pond & 1 & 1\\
\rowcolor{gray!6}  Turbid & 3 & 3\\
Wedge Creek & 1 & 3\\
\bottomrule
\end{tabular}
\end{capctable}

\end{document}
